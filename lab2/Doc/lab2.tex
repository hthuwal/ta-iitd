\documentclass{article}
\usepackage{hyperref}

\title{Lab - 2}
\date{Week of Jan. 14, 2018}

\begin{document}

\maketitle

Questions 1-3 should be tried in the OCaml toplevel.
\begin{enumerate}

\item Set the variables x, y and z to any \emph{positive} integers of your choice and compute the following expressions:
	\begin{enumerate}
		\item $x^3 + 2xyz^2 − yz + 1$
		\item Cube root of $xyz$
		\item $ln (x + \sqrt{(x^2 + 1)})$
		\item The sine function in OCaml takes in the angle in radians. However, we would like to supply the angle in degrees. Therefore, there is a need to convert the angle in degrees to the angle in radians. Write the expression to do so. Use the following value of $\pi$ discussed in class: $4atan(1)$ and the formula for converting degrees into radians as: 
		$radians$ = $degrees * \frac{\pi}{180}$.
	\end{enumerate}
Use the following web page (the section on floating point arithmetic) for help: \url{https://caml.inria.fr/pub/docs/manual-ocaml/libref/Pervasives.html}

\item Write the following functions for each of the above:
	\begin{enumerate}
		\item {\tt poly} (takes in 3 integers, returns one integer)
		\item {\tt mcuberoot} (takes in 3 floats, returns one float)
		\item {\tt nlog} (takes in 1 float, returns 1 float)
		\item {\tt degrees\_to\_radians} (takes in 1 float, returns 1 float)
	\end{enumerate}
\item We would like to ensure that the inputs to the functions above are correctly bound. Write input validation (or correction) functions for each of the above:
	\begin{enumerate}
		\item {\tt check\_poly}: takes in 3 integers provided as input, returns true only if each integer is positive. If any of the integers is negative, then it returns false.
		\item {\tt check\_mcuberoot}: takes in 3 floats, returns true only if there product (xyz) is positive, otherwise, returns false
		\item {\tt check\_nlog}: takes in 1 float, determine for yourself when this function should return true.
		\item {\tt check\_degrees\_to\_radians}: takes in 1 float, returns a value (in degrees) that is in between 0 and 360. NOTE: If the parameter is 361 degrees, then this function returns 1 degree, if the value is -1 degree, then it returns 359 degrees, etc.
	\end{enumerate}
\item Source files
	\begin{enumerate}
		\item Write function {\tt poly} and the corresponding input validation function in the file {\tt poly.ml}. In the poly function first call the {\tt check\_poly} (validation function) to ensure that the inputs are valid. If the inputs are valid then return the value of the polynomial else return {\tt-1.0}.
		\item Write a main function that calls function {\tt poly} with the following values: $x = 2, y = 3, z = 4$ and prints out the value returned by the function.
		\item Compile {\tt poly.ml} into the executable {\tt poly}
		\item Run {\tt poly}. Ensure that the behaviour of the program is as expected.
	\end{enumerate}
\item Remove the main function from the {\tt poly.ml}
	\begin{enumerate}
		\item Load the functions in {\tt poly.ml} into the toplevel.
		\item Test the functionality with the following sets of values:
			$$x = -5, y = 3, z = 8$$
			$$x = 2, y = 1.2, z = 0$$
			$$x = 5, y = -3, z = 2$$
		\item Think of more test cases. What errors can occur?
	\end{enumerate}

\item Repeat steps 4 and 5 for the other functions in 2).

\item Prepare and submit the following files for submission on moodle:

	\begin{enumerate}
		\item {\tt poly.ml} (which implements 2a and 3a as specified in 4a)
		\item {\tt mcuberoot.ml} (which implements 2b and 3b)
		\item {\tt nlog.ml} (which implements 2c and 3c)
		\item {\tt degrees\_to\_radians.ml} (which implements 2d and 3d)
	\end{enumerate}
	The deadline for submission on moodle is {\tt Sunday, 21 January 2018, 11:59 PM}

\end{enumerate}

\end{document}